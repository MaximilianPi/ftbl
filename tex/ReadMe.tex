\documentclass[10pt,a4paper,notitlepage,onecolumn]{article}
%----------Style Section



\usepackage[latin1]{inputenc}

\usepackage[english]{babel} %English Corrections

\usepackage[justification=RaggedRight,singlelinecheck=off]{caption} %Left aligns all captions
\captionsetup[longtable]{format=hang,justification=raggedright,singlelinecheck=false} %should left align all longtable catpions as well


\usepackage{array}

\usepackage{ booktabs, longtable}
\setlength{\LTleft}{0pt} %Left align all longtables
%\usepackage[plainnat]{natbib}


\usepackage[pdftex]{graphicx}
\usepackage{fancyhdr} %necessary for title page. why ever... :-)
\usepackage{epstopdf}
\usepackage{float}

\usepackage[top=3.5cm, bottom=3cm, inner=4.0cm, outer=3.0cm]{geometry} %Change margins
%\usepackage{layout}

%\usepackage{showframe} %Package allows to plot margins


\newcommand{\ra}[1]{\renewcommand{\arraystretch}{#1}}
\newcommand\mpar[1]{\marginpar {\flushleft\small #1}}
\setlength{\marginparwidth}{3cm}


\begin{document}


\begin{raggedright}

\subsection*{Besprechung Ver�nderungsinventur November 2017}
22.11.2017, 10:00 \\
Meinrad Abegg (ma), Jonas Stillhard (js) 
\bottomrule

\subsubsection*{Test Wiederauffindbarkeit PF Oktober 2017}
Im Rahmen der Ukrainereise von Peter, Martina, Roksolana und Jonas wurden 10 PF mit einem Garmin GPS angelaufen. Als weiteres Hilfsmaterial standen die bei der Aufnahme 2010 auf den PF gemachten Fotos sowie ein Kroki und eine Liste der Probeb�ume zur Verf�gung. Alle PF konnten relativ leicht wiedergefunden werden. \\
Tendenziell konnten die mit einem Trimble Geo XH eingemessenen PF ein wenig leichter gefunden werden, da uns das GPS n�her ans PF-Zentrum gebracht hat. \\
F�r das Wiederauffinden wurden maximal 25 Minuten, im Durchschnitt wohl ca. 10 Minuten gebraucht. 

\subsubsection*{Fahrplan 2018/19}
Grober Ablauf f�r die n�chsten 1.5 Jahre inkl. Deadlines.

\begin{longtable}{p{9.5cm} p{4.0cm}}
\caption{Fahrplan }
\label{tab:to do}\\

\toprule
\textbf{Schritt} & \textbf{Due Date} \\
\midrule
\endfirsthead
\toprule
\textbf{Schritt} & \textbf{Due Date} \\
\midrule
\endhead
\midrule
\multicolumn{3}{c}{{\textit{continued}}}\\ %Text that appears at the end of the table before breaking to the next page
\bottomrule
\endfoot
\bottomrule
\endlastfoot
Auswertungskonzept Ver�nderungsinventur & 8. Januar 2018\\
Definition Merkmalskatalog soweit m�glich & 10. Januar 2018 \\
Beschaffung Material soweit m�glich & 31.M�rz 2018\\
Datenvorgaben f�r Pilotinventur & Mitte April 2018\\
Draft Aufnahmeanleitung & Mitte April 2018 \\
Pilotinventur & Mai / Juni 2018 \\
Aufnahmedesign, Aufnahmefahrplan, Anzahl Gruppen & Oktober 2018 \\
Ausschreibung Studenten & November 2018 \\
Bewerbungsgespr�che Studenten & Januar/Februar 2018 \\
Vorbereitung Inventur, ukrainischer Co-Leiter an der WSL & M�rz 2019
Hauptinventur & Juli / August 2019 \\
\end{longtable}

\textit{Auswertungskonzept:} F�r die Definition des Merkmalkatalogs soll ein Auswertungkonzept f�r einzelne Zielgr�ssen erarbeitet werden. js bereitet bis zum 25.11 ein Mail an das Projektteam vor, in welchem alle Mitglieder aufgefordert werden, Ihre gew�nschten Zielgr�ssen etc. bis zum 8.1 zu liefern.  \\


\subsubsection*{Beschaffung Material}

F�r das Projekt ist es wichtig, dass bis zum 31. M�rz 2018 ca. 200k Franken der Projektsumme ausgegeben sind. Es stellt sich die Frage, was wir bereits jetzt beschaffen k�nnen. js kl�rt mit pb ab, wieviel Geld unser Teilprojekt ausgeben soll.



\subsubsection*{Vorbereitungen Pilotinventur 2018}
\textit{Ziele Pilotinventur:} Zeitmanagement, Aufnahme Cluster vs. einzelne PF. \\
\textit{Personal:} js wird Pilotinventur durchf�hren, in Zusammenarbeit mit ukrainischem Co-Leiter.\\

\subsubsection*{Methodisches}
Aufnahmemekrmale werden im Rahmen des Auswertungskonzept definiert. \\ 
Ver�nderungs-Merkmale: Wird im Rahmen Definition Aufnahmekatalog besprochen.\\


\subsubsection*{Ver�nderungsinventur 2019}
Ukrainischer Co-Leiter: Es spricht nichts gegen Mykola Korol. js wird in n�chster Zeit (bis 31. M�rz) versuchen, eine Vereinbarung mit Mykola aufzusetzen. 

\subsubsection*{Guest Stays im Rahmen Ver�nderungsinventur}
F�r M. Korol ist ein Guest Stay im Fr�hjahr 2019 f�r die Vorbereitung der Hauptinventur vorgesehen. \\
Falls Mykola in einem weiteren Call einen Antrag f�r Teilnahme am Scientific Exchange einreicht und dieser bewilligt wird, so werden wir seien Aufenthalt an der WSL aus dem Budget des Teilprojekts bezahlen. \\
Die verbleibenden 4 Monaten sollen f�r einen ukrainischen Masterstudenten / Doktoranden nach der Inventur eingesetzt werden. 



\subsubsection*{Homepage}
\textit{Die Migration hat der nicht gut getan.}



\begin{longtable}{p{7.5cm} p{2.0cm}  p{2.0cm}}
\caption{to do }
\toprule
\textbf{to do} & \textbf{responsible} & \textbf{due date} \\
\midrule
\endfirsthead
\toprule
\textbf{to do} & \textbf{responsible} & \textbf{due to date} \\
\midrule
\endhead
\midrule
\multicolumn{3}{c}{{\textit{continued}}}\\ %Text that appears at the end of the table before breaking to the next page
\bottomrule
\endfoot
\bottomrule
\endlastfoot
Homepage. Check Links to project etc. & js & 31.11.2017 \\
Probefl�chennetz & js/ma & 31.10.2017\\
Brigitte fragen bzgl Vorgehen letzte Inventur Ausf�lle & js & 22.12.2017\\
Kl�ren Forschungsziele und Anspr�che an Genauigkeit GPS-Messungen & ma/js & 22.12.2017\\
Abkl�ren pb Erwartungen Projekt Ausgaben Teilprojekt & js & 7.12.17 \\
Aufnahmekatalog kl�ren & ma/js & 10.1.2018 \\
Definition Auswertungen / Paperideen / Guest Stays etc. & js/ma & 31.3.2018 \\



\end{longtable}

\textbf{N�chste Termine:}\\
10.1.18, 13:00-17:00: Besprechung Merkmalskatalog \\



\end{raggedright}


\end{document}
